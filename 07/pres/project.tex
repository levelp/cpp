\section{Java EE: создание сайта}

\begin{frame}[t]{ЯДРО СИСТЕМЫ / Предметная область}

  \begin{itemize}
    \item Объекты, их свойства
    \item ..и действия с ними (бизнес-процессы)
  \end{itemize}

  ОО Моделирование / абстрагирование

Технологии: 
  \begin{itemize}
    \item Java-классы (изучаем Java SE) 
    \item Unit-тестирование (JUnit)
  \end{itemize}
\end{frame}


\begin{frame}[t]{Domain-driven design, DDD}


Предметно-ориентированное проектирование ()

это набор принципов и схем, помогающих разработчикам создавать изящные системы объектов. При правильном применении оно приводит к созданию программных абстракций, которые называются моделями предметных областей. В эти модели входит сложная бизнес-логика, устраняющая промежуток между реальными условиями области применения продукта и кодом.[1]

Предметно-ориентированное проектирование не является какой-либо конкретной технологией или методологией. DDD — это набор правил, которые позволяют принимать правильные проектные решения. Данный подход позволяет значительно ускорить процесс проектирования программного обеспечения в незнакомой предметной области.


\lstinputlisting[language=Java]{ex.java}


База данных / хранилище.
 Как с ним работать, запросы, SQL

Технологии:
 * Hibernate + Spring ORM
 * PostgreSQL (или MySQL)  

Интерфейс 
 * Функциональность
  * Что пользователь может делать и как
   * Авторизация (регистрация). 
     * Роли (виды) пользователей и т.д.
     * Spring Security 
   * Ввод данных (в зависимости от роли)
     * Spring + виджеты 
   * Поиск по данным (в том числе отчёты) + просмотр
 * GUI / Дизайн (как это выглядит)
   * GWT или Bootstrap / jQuery UI 
   * CSS + HTML5 + jQuery. 
    
Взаимодействие с другими системами
 * Web-сервисы
 * REST-интерфейс для Android-приложений
 * Импорт-экспорт
 


\end{frame}
