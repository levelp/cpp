\section{Языки высокого и низкого уровня}
\subsection{Уровень языка программирования (ЯП)}

\begin{frame}[t]{Уровень языка программирования (ЯП)}
  \begin{itemize}
    \item \textbf{Высокоуровневый ЯП} --- \textbf{скорость} и \textbf{удобство} разработки (удобство программиста) в том числе за счёт снижения 
    эффективности использования памяти и процессорного времени. 
    Ближе к естественному языку.

    \item \textbf{Язык низкого уровня} --- близок к программированию в машинных кодах (или на ассемблере)
      используемого реального или виртуального (Java, .NET) процессора
  \end{itemize}
\end{frame}

\subsection{Особенности языков высокого уровня}

\begin{frame}[t]{Язык высокого уровня}

   Основная черта - \textbf{абстракция}, то есть введение смысловых конструкций, 
   кратко описывающих такие структуры данных и операции над ними, 
   описания которых в машинном коде (или низкоуровневом ЯП) длинны и сложны для понимания.

\end{frame}

\begin{frame}[t]{Основные <<+>> ЯП низкого уровня}
  \begin{itemize}
    \item эффективное использование процессорного $t$ и памяти.
    \item часто язык низкого уровня позволяет обратиться к ресурсам, недоступным из языка высокого уровня.
    \item размер исполняемого файла готовой программы получается, как правило, меньше.
  \end{itemize}
\end{frame}


\begin{frame}[t]{История языка C}

  C (Си) --- компилируемый статически типизированный ЯП общего назначения,
  разработанный в 1969-1973 годах сотрудником Bell Labs Деннисом Ритчи как развитие языка B.



\end{frame}

\begin{frame}[t]{Возникновение названия языка C}

  \begin{itemize}
  \item 1964 - \textbf{APL} - назван по книге \textbf{A Programming Language} --- ЯП, 
     оптимизированный для работы с массивами, предшественник современных вычислительных сред (MATLAB), 
   использует функциональную парадигму программирования.    
   Разработан Кеном Айверсоном, преподававшим тогда в Гарвардском университете, 
   в качестве системы обозначений для описания вычислений. 
   В 1957 вышла его книга <<A Program Language>>, в которой был описан \textbf{APL}. 
   
\item 1966 - \textbf{BCPL} (Basic Combined Programming Language) --- ЯП,
разработанный Мартином Ричардсом, в Кембриджском университете. 
Изначально предназначался для написания компиляторов для других языков.
  Сейчас BCPL практически не используется, но в своё время он был очень важен из-за хорошей портируемости. 
  Урезанная версия языка с несколько изменённым синтаксисом стала ЯП \textbf{B}, который оказал сильное влияние на \textbf{C}. 
  \end{itemize}
\end{frame}

\begin{frame}[t]{Возникновение названия языка C - 2}
  \begin{itemize}
  
  \item 1969 - Язык \textbf{B} --- интерпретируемый ЯП, разработанный в AT\&T Bell Telephone Laboratories. 
    Является потомком языка BCPL и непосредственным предшественником \textbf{C}. 
    \textbf{B} был в основном произведением Кена Томпсона при содействии Денниса Ритчи и был опубликован в 1969 году.
    
  \item 1972 - \textbf{C} --- название языка является третьей буквой алфавита (намекает что \textbf{C} более совершеннный чем \textbf{B})
  
Успех \textbf{C} в основном связан с тем, что на нём была написана значительная часть операционной системы UNIX, 
которая в итоге приобрела очень большую популярность. 
В связи с её распространённостю, и с тем, что объём ОС измеряется в миллионах строк кода 
(для примера, в последних версиях Linux содержится более 10000000 строк кода), 
задача о переписывании UNIX на другой язык становиться практически невыполнимой 
(также следует учитывать тот факт, что при ручном переписывании неизбежно возникнут ошибки, 
что существенно снизит стабильность работы, а при переводе с использованием программных средств пострадает производительность кода).
Кроме того, \textbf{C}, будучи приближённым к аппаратной реализации компьютера позволяет выжать из него намного больше, 
чем многие другие языки программирования. 
Это обстоятельство показывает бессмысленность перевода UNIX на другой язык. 
Таким образом, если другие языки программирования могут исчезнуть с течением времени, уступив дорогу новым технологиям, 
то \textbf{C} будет жить, пока живёт UNIX. То есть пока существуют компьютеры в том виде, в котором мы их себе представляем.

Первая книга, посвящённая \textbf{C} была написана Керниганом и Ритчи в 1978 году и вышла в свет под названием 
<<Язык программирования Си>>. Эта книга, в среде программистов более известная как <<K\&R>>, стала неофициальным стандартом \textbf{C}.  



  \end{itemize}

\end{frame}


\begin{frame}[t]{C11}

  8 декабря 2011 опубликован новый стандарт для языка С (ISO/IEC 9899:2011)

\end{frame}
