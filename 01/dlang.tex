\section{Язык D - http://dlang.org}
\subsection{Язык D}

\begin{frame}[t,fragile]{Развитие C - язык D - http://dlang.org}
  \textbf{D} - объектно-ориентированный, императивный, мультипарадигмальный язык программирования, 
  созданный Уолтером Брайтом из компании Digital Mars. 
  Изначально был задуман как реинжиниринг языка C++, однако, несмотря на значительное влияние С++, не является его вариантом. 
  В D были заново реализованы некоторые свойства C++, 
  также язык испытал влияние концепций из других языков программирования, таких как Java, Python, Ruby, C\# и Eiffel.

\begin{lstlisting}
void main() {
  // Define an array of numbers, double[].
  auto arr = [ 1, 2, 3.14, 5.1, 6 ];
  auto dictionary = [ "one" : 1, "two" : 2, "three" : 3 ];
  // Calls the min function defined below
  auto x = min(arr[0], dictionary["two"]);
}
// Type deduction works for function results
auto min(T1, T2)(T1 lhs, T2 rhs) {
  return rhs < lhs ? rhs : lhs;
}
\end{lstlisting}


\end{frame}
