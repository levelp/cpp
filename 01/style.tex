\section{Стиль оформления кода программы}
\subsection{Зачем придерживаться одного стиля?}

\begin{frame}[t]{Зачем придерживаться одного стиля?}

Как правило, программист больше \textbf{читает} код программ, чем \textbf{пишет} новый.

Одним из важнейших факторов, влияющих на способность программы к развитию, 
является \textbf{понятность} кода. 

Одним из существенных факторов понимаемости программы, в свою очередь, 
является информативность исходного текста. 
Если исходный текст не является хорошо читаемым, 
то есть написан без соблюдения определенного стиля и системы и представляет 
собой <<мешанину>> операторов и знаков препинания, то вносить изменения в него очень сложно даже автору. Рассмотрим ряд требований и рекомендаций, позволяющих выработать хороший стиль оформления программ, повышающий ее информативность.


\end{frame}
                                          
\subsection{Рекомендации}

\begin{frame}[t]{Рекомендации}

    \textbf{1. Допускаются любые нарушения рекомендаций, если это улучшает читаемость.}

Основная цель рекомендаций --- улучшение читаемости и, следовательно, ясности и лёгкости поддержки, 
а также общего качества кода. 
Невозможно дать рекомендации на все случаи жизни, поэтому программист должен мыслить гибко.

\end{frame}

\begin{frame}[t,fragile]{Соглашения об именовании}

Имена, представляющие типы, должны быть обязательно написаны в смешанном регистре, начиная с верхнего (Camel) --- 
общая практика в сообществе разработчиков C++

         \begin{lstlisting}
class Line{ ... }; struct SavingsAccount{ ... }  
         \end{lstlisting}

Имена переменных должны быть записаны в смешанном регистре, начиная с нижнего.

         \begin{lstlisting}
int line; string savingsAccount
         \end{lstlisting}

Именованные константы (включая значения перечислений) должны быть записаны в верхнем регистре с нижним подчёркиванием в качестве разделителя.

\begin{lstlisting}
const int MAX_ITERATIONS = 100; const Color COLOR_RED = ...;
\end{lstlisting}

\end{frame}

\begin{frame}[t,fragile]{Соглашения об именовании}

Названия методов и функций должны быть глаголами, быть записанными в смешанном регистре и начинаться с нижнего: 

\begin{lstlisting}
getName(), computeTotalWidth()
\end{lstlisting}

Названия пространств имён следует записывать в нижнем регистре 

\begin{lstlisting}
model::analyzer, io::iomanager, common::math::geometry
\end{lstlisting}

Следует называть имена типов в шаблонах одной заглавной буквой 

\begin{lstlisting}
template<class T> ... template<class C, class D> 
\end{lstlisting}

\end{frame}
