\section{SQL (на примере sqlite), QtSql}

\begin{frame}[t]{Язык SQL}

\textbf{Structured Query Language --- <<структурированный язык запросов>>} --- 
формальный непроцедурный язык программирования,
применяемый для создания, модификации и управления данными в произвольной реляционной БД,
управляемой соответствующей системой управления БД (СУБД). SQL основывается на исчислении кортежей.

\textbf{Data Definition Language (DDL)} --- язык описания структуры БД.

\end{frame}


\begin{frame}[t]{Язык SQL: SELECT, INSERT, UPDATE, DELETE}
  \begin{itemize}
    \item Язык SQL: SELECT, INSERT, UPDATE, DELETE
Идея задачи --- Глеб Евстропов
    \item Подготовка тестов --- Павел Кунявский, Глеб Евстропов
    \item Разбор задачи --- Глеб Евстропов
  \end{itemize}
\end{frame}

\subsection{Формулировка задачи}

\begin{frame}[t]{Математическая формулировка}
  Входные данные:
  \begin{itemize}
    \item Количество элементов последовательности $n$
    \item Диапазон цветов $k$
    \item Последовательность цветов $c_i$
  \end{itemize}
  Результат:
  \begin{itemize}
    \item Количество подотрезков исходной последовательности, на которых какой то цвет встречается больше раз чем все остальные вместе взятые
  \end{itemize}
\end{frame}

\subsection{Решение}

\begin{frame}[t]{$O(n^3 \cdot k)$}
  \begin{itemize}
    \uncover<1->{\item Решим задачу отдельно для каждого подотрезка}
    \uncover<2->{\item Отдельно проверим каждый цвет на преобладание на выбранном отрезке}
    \uncover<3->{\item Пусть цвет встречается $cnt$ раз на отрезке $(i, j)$, тогда цвет преобладает на этом отрезке если $cnt \cdot 2 > (j - i + 1)$}
    \uncover<4->{\item Итоговая сложность $O(n^3 \cdot k) \le$ кол-во отрезков $\times$ кол-во цветов $\times$ максимальную длину отрезка}
  \end{itemize}
\end{frame}

\begin{frame}[t]{$O(n^3 \cdot \log n)$ и $O(n^3)$}
  \begin{itemize}
    \uncover<1->{\item Как и до этого решаем задачу отдельно для каждого отрезка}
    \uncover<2->{\item Выделить самый часто встречающийся элемент можно с помощью сортировки элементов на подотрезке - $O(n^3 \log n)$, а затем проверять его на преобладание не этом отрезке}
    \uncover<3->{\item В отдельном массиве посчитаем частоты встречаемости всех элементов на подотрезке и за линию выберем максимум, получив сложность $O(n^3)$}
  \end{itemize}
\end{frame}

\begin{frame}[t]{$O(n^2 k)$ и $O(n^2)$}
  \begin{itemize}
    \uncover<1->{\item Теперь будем фиксировать только левую границу, то есть при $i = const$ вычислять ответ для всех $j \ge i$}
    \uncover<2->{\item Поддерживаем массив с количествами цветов на текущем отрезке. Очевидно, что при сдвиге на единицу вправу нужно изменить только одно значение этого массива}
    \uncover<3->{\item Если для выбора максимума по-прежнему линейно пробегать весь массив, то получаем $O(n^2 \cdot k)$. Однако при сдвиге вправо максимум можно пересчитывать за $O(1)$ проверяя цвет нового элемента}
    \uncover<4->{\item Таким образом, каждый сдвиг выполняется за $O(1)$ и итоговая сложность $O(n^2)$}
  \end{itemize}
\end{frame}

\begin{frame}[t]{$O(nk \log n)$}
  \begin{itemize}
    \uncover<1->{\item Легко заметить, что так как условие преобладания строгое, то на отрезке не может быть двух и более преобладающих цветов.}
    \uncover<2->{\item Из этого соображения следует, что по каждому цвету задачу можно решать независимо.}
    \uncover<3->{\item Если зафиксировать цвет c и заменить на $+1$ все элементы данного цвета и на $-1$ все остальные, то ответом для данного цвета является кол-во отрезков с положительной суммой.}
    \uncover<4->{\item Если перейти к последовательности частичных сумм (сумм на префиксах), то для каждого элемента требуется сказать кол-во элементов левее и меньше его. Это стандартная задача решающаяся за сложность $O(n \log n)$}
    \uncover<5->{\item Итоговая сложность решения: $O(kn \log n)$}
  \end{itemize}
\end{frame}

\begin{frame}[t]{$O(n \cdot k)$}
  \begin{itemize}
    \uncover<1->{\item Научимся решать задачу для одного цвета за $O(n)$. Как было сказано, требуется в последовательности частичных сумм для каждого элемента найти количество меньших его и расположенных левее.}
    \uncover<2->{\item Заметим, что так как два соседних элемента отличаются ровно на $1$, то между двумя соседними элементами одного значения, все промежуточные либо строго больше, либо строго меньше.}
    \uncover<3->{\item Отсюда следует, что ответ для фиксированной позиции легко пересчитывается через предыдущую позицию с таким же значением частичной суммы.}
  \end{itemize}
\end{frame}

\begin{frame}[t]{$O(n^{1.5} \cdot \log n)$ и $O(n^{1.5})$}
  \begin{itemize}
    \uncover<1->{\item Заметим, что если кол-во элементов цвета $c$ равно $cnt$, то он может преобладать на отрезках длины не более $2cnt - 1$.}
    \uncover<2->{\item Разделим цвета на две группы: встречаемость которых больше либо равна $n^{0.5}$ и наоборот.}
    \uncover<3->{\item Для первой группы мы умеем решать задачу за $O(n)$ для одного цвета, но количество цветов в этой группе не превосходит $n^{0.5}$.}
    \uncover<4->{\item Для второй группы достаточно рассмотреть все отрезки, длина которых не превосходит $2 n^{0.5}$.}
    \uncover<5->{\item Обе части, а, следовательно, и все решение, имеют асимптотику $O(n^{1.5})$. $O(n^{1.5} \log n)$ при использовании стандартных решений для решения первой группы.}
  \end{itemize}
\end{frame}

\begin{frame}[t]{$O(n \log n)$ и $O(n)$}
  \begin{itemize}
    \uncover<1->{\item Рассмотрим следующие три примера: все $+1$ образуют непрерывный отрезок, $+1$ расположены вперемешку с $-1$ на отрезке длины $2 cnt$ и $+1$ образую два сильно разведенных отрезка.}
    \uncover<2->{\item Легко заметить, что в первых двух случаях достаточно отойти не более чем на $cnt$ вправо и на $cnt$ влево и решить за линейное решение задачу на данном подотрезке. В третьем случае задача разбивается на $2$ отрезка, оставаясь суммарно линейной по $cnt$.}
    \uncover<3->{\item Делаем предположение, что задачу не нужно решать на всех подотрезках $(1, n)$, а только на на некотором множестве непересекающихся отрезков $(l_1, r_1), (l_2, r_2) ... (l_m, r_m)$, таких что $\sum_{i=1}^{m}(r_i-l_i+1) = O(cnt)$.}
  \end{itemize}
\end{frame}

\begin{frame}[t]{$O(n \log n)$ и $O(n)$}
  \begin{itemize}
    \uncover<1->{\item Утверждение: искомое разбиение находится как объединение отрезков следующего вида: для каждого $i$ рассмотрим минимальное $j$ такое что сумма на отрезке $(j, i)$ положительна, и максимальное $j$ с таким же условием.}
    \uncover<2->{\item Жюри располагает строгим доказательством этого факта и готово изложить его по просьбе участников, здесь же оно не приводится для экономии места и времени, а так же является простым и полезным упражнением.}
    \uncover<3->{\item Итого: решим задачу независимо по цветам, для каждого цвета разобъем на отрезки суммарной длины $O(cnt)$ и решим на них. Итоговая сложность $O(n)$. }
  \end{itemize}
\end{frame}
