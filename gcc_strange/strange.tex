\documentclass[mathserif,utf8,xcolor=table,11pt]{beamer}

\usepackage[T2A]{fontenc}
\usepackage[utf8]{inputenc}
\usepackage[english,russian]{babel}
\usepackage{graphicx}
\usepackage{ulem}
\usepackage{textpos}
\usepackage{hyperref}
%\usepackage{listings}
\usepackage{listingsutf8}
%\usepackage{minted}
\usepackage{tikz}
\usetikzlibrary{mindmap,shadows}
%\usepackage[hidelinks,pdfencoding=auto]{hyperref}
\usepackage{verbatim}

\definecolor{dkgreen}{rgb}{0,0.6,0}
\definecolor{gray}{rgb}{0.5,0.5,0.5}
\definecolor{mauve}{rgb}{0.58,0,0.82}

\lstset{frame=none,
  language=C++,
%  aboveskip=3mm,
%  belowskip=3mm,
  showstringspaces=false,
  columns=flexible,
  basicstyle={\scriptsize\ttfamily},
  numbers=none,
  numberstyle=\tiny\color{gray},
  keywordstyle=\color{blue},
  commentstyle=\color{dkgreen},
  stringstyle=\color{mauve},
  breaklines=true,
%  breakatwhitespace=true,
  tabsize=2,
 % escapeinside={\%*}{*)},
 % inputencoding=utf8,
  inputencoding=utf8/utf8,
  texcl=true,
}

\mode<presentation>
{
        \usetheme{Antibes}
        \setbeamercovered{transparent}
}

\title{<<Странные>> моменты в С/C++: проверьте себя}
\institute{Степулёнок Денис Олегович}
\date[август 2016]{17 августа 2016 года}
\subject{Занятие 1}

\newcommand{\hl}{\only{\cellcolor{yellow}}}
\renewcommand{\le}{\leqslant}
\renewcommand{\ge}{\geqslant}
\setlength{\arrayrulewidth}{1pt}

\begin{document}

\section{Индексация массивов}
\subsection{sol01.cpp}
\begin{frame}[t]{Массивы: что выведет эта программа?}
  \lstinputlisting[language=C++]{sol01.cpp}
  \lstinputlisting[language={}]{sol01.txt}

Почему $6$? Как работает индексация в C/C++?

Все примеры можно посмотреть и скачать:
\url{https://github.com/levelp/cpp/tree/master/gcc_strange}

\end{frame}


\section{Переполнения}
\subsection{sol02.cpp}
\begin{frame}[t]{А как насчёт переполнений? Что вы знаете о них?}
  \lstinputlisting[language=C++]{sol02.cpp}
\end{frame}

\subsection{sol05.cpp}
\begin{frame}[t]{Переполнения: закончится ли когда-либо цикл?}
  \lstinputlisting[language=C++]{sol05.cpp}

9
81
729
6561
59049
531441
4782969
43046721
387420489
-808182895
1316288537
-1038305055
-754810903
1796636465
\end{frame}

\section{Приоритет операций}
\subsection{sol03.cpp}
\begin{frame}[t]{<<Странный>> приоритет операций в стандарте C++}
  \lstinputlisting[language=C++]{sol03.cpp}

  \lstinputlisting[language={}]{sol03.txt}
\end{frame}

\section{sizeof}
\subsection{sol04.cpp}
\begin{frame}[t]{Коварный <<sizeof>>}
  \lstinputlisting[language=C++]{sol04.cpp}

 % \lstinputlisting[language={}]{sol04.txt}
\end{frame}

\section{1 <= 2 <= 3}
\subsection{sol06.cpp}
\begin{frame}[t]{Проверяем ассоциативность}
  \lstinputlisting[language=C++]{sol06.cpp}
\end{frame}

\section{Приоритет операций}
\subsection{sol07.cpp}
\begin{frame}[t]{Порядок ввода и порядок вывода}
  \lstinputlisting[language=C++]{sol07.cpp}
\end{frame}

\section{a++ ++a}
\subsection{sol08.cpp}
\begin{frame}[t]{Преинкремент и постинкремент}
  \lstinputlisting[language=C++]{sol08.cpp}
  \lstinputlisting[language={}]{sol08.txt}
 В GCC ответ 553, как он получается?
\end{frame}

% Не знаю что в этом примере не так
%\section{Приоритеты}
%\subsection{sol09.cpp}
%\begin{frame}[t]{Приоритет операций. Ставим скобки!}
% Приоритет операций неочевиден, 
% результат вычислений непредсказуем.
%
%  \lstinputlisting[language=C++]{sol09.cpp}
%\end{frame}

\section{Фокусы с map}
\subsection{sol10.cpp}
\begin{frame}[t]{Фокусы с map}
  \lstinputlisting[language=C++]{sol10.cpp}
  \lstinputlisting[language={}]{sol10.txt}
\end{frame}

\section{Приоритеты}
\subsection{sol11.cpp}
\begin{frame}[t]{Проверяем ассоциативность}
  \lstinputlisting[language=C++]{sol11.cpp}
 % \lstinputlisting[language={}]{sol11.txt}
\end{frame}

\section{sizeof}
\subsection{sol12.cpp}
\begin{frame}[t]{Проверяем ассоциативность}
  \lstinputlisting[language=C++]{sol12.cpp}
 % \lstinputlisting[language={}]{sol12.txt}
\end{frame}

\section{sizeof}
\subsection{sol13.cpp}
\begin{frame}[t]{Проверяем ассоциативность}
  \lstinputlisting[language=C++]{sol13.cpp}
 % \lstinputlisting[language={}]{sol13.txt}
\end{frame}

\section{sizeof}
\subsection{sol14.cpp}
\begin{frame}[t]{Проверяем ассоциативность}
  \lstinputlisting[language=C++]{sol14.cpp}
%  \lstinputlisting[language={}]{sol14.txt}
\end{frame}

\section{sizeof}
\subsection{sol15.cpp}
\begin{frame}[t]{Проверяем ассоциативность}
  \lstinputlisting[language=C++]{sol15.cpp}
  \lstinputlisting[language={}]{sol15.txt}
\end{frame}

\section{Динамическая память}
\subsection{sol16.cpp}
\begin{frame}[t]{Фокусы с вектором}
  \lstinputlisting[language=C++]{sol16.cpp}
\end{frame}

\section{Приоритеты}
\subsection{sol17.cpp}
\begin{frame}[t]{Тернарный оператор и приоритеты операций}
  \lstinputlisting[language=C++]{sol17.cpp}
  \lstinputlisting[language={}]{sol17.txt}

Таблица приоритетов операций в C++:
\url{http://en.cppreference.com/w/cpp/language/operator_precedence}
\end{frame}

\section{Переполнения}
\subsection{sol18.cpp}
\begin{frame}[t]{Переполнения}
  \lstinputlisting[language=C++]{sol18.cpp}
\end{frame}


\end{document}
