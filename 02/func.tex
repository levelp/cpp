\section{Функции в C++}

\subsection{Рекурсия}
\begin{frame}[t,fragile]{Рекурсия - функция Факториал}
Итеративное вычисление факториала: от большего к меньшему $i = n, n-1, n-2 ... 2$
\begin{lstlisting}
long long factorial(int n) {
  long long r = 1; // Результат вычислений
  for(int i = n; i > 1; --i)
    r *= i; // Умножаем на i
  return r;
}
\end{lstlisting}

Цикл от меньшего к большему:
\begin{lstlisting}
  for(int i = 2; i <= n; ++i)
    r *= i;
\end{lstlisting}

\textbf{Рекурсия} --- когда функция вызывает сама себя.

Рекурсивный способ вычисления факториала:
\begin{lstlisting}
long long f(int n) {
  if(n <= 1)
    return 1; // Возвращаем 1
  else
    return n * f(n - 1); // Рекурсивный вызов
}
\end{lstlisting}

\end{frame}

\subsection{Передача параметров}

\begin{frame}[t,fragile]{Аргумент: значение, ссылка, указатель}
\begin{lstlisting}
// По значению - by value - копируется значение
// Внутри функции возникает новая переменная i, при изменении которой
// <<основная>> i никак не меняется
void f1(int  i) { i++;     cout << "i = " << i << endl;  }
// По ссылке - by reference - только в C++, но не в C
// i внутри функции - новое имя (алиас, alias) для внешней переменной
void f2(int& i) { i++;     cout << "i = " << i << endl;  }
// По указателю - by pointer
void f3(int* i) { (*i)++;  cout << "i = " << *i << endl; }

int main() { // Основная программа
  int i = 3; // <<Основная>> переменная i
    // Сначала она равна 3
  f1(i);  cout << "i = " << i << endl; // i = 3 по-прежнему
  f2(i);  cout << "i = " << i << endl; // i увеличилась
  f3(&i); cout << "i = " << i << endl; // i ещё раз увеличилась
  return 0;
}
\end{lstlisting}
\end{frame}

\begin{frame}[t,fragile]{Аргумент: по значению - <<обычная>> переменная}
\begin{lstlisting}
// По значению - by value - копируется значение
// Внутри функции возникает новая переменная i, при изменении которой
// "основная" i никак не меняется
void f1(int  i) { 
  i++;    
  cout << "i = " << i << endl; 
}
\end{lstlisting}

Внутри функции появляется новая переменная с тем значением, которое было передано внутрь функции.

Эту переменную можно использовать и менять как угодно и это никак не отразится на
переменной во внешней (вызывающей) программе.

Можно сказать, что эта переменная только для внутреннего использования.  

\end{frame}

\begin{frame}[t,fragile]{Аргумент: по ссылке - только в С++, не в C}
\begin{lstlisting}
// По ссылке - by reference
// i внутри функции - новое имя (алиас, alias) для внешней переменной
void f2(int& i) { 
  i++;     
  cout << "i = " << i << endl;  
}
\end{lstlisting}

Внутри функции появляется переменная, которая является другим именем для
внешней переменной.

Если мы поменяем значение переменной внутри функции, изменится и внешняя переменная.

Это можно использовать чтобы возвращать значения из функции. 
Или если функция должна менять значения в основной программе.
Например, есть какой-то глобальный счётчик и функция должна его увеличивать и ещё что-то.
\end{frame}

\begin{frame}[t,fragile]{Аргумент: по указателю}
\begin{lstlisting}
// По указателю - by pointer
void f3(int* i) { 
  (*i)++;  
  cout << "i = " << *i << endl; 
}
\end{lstlisting}

При этом внутри функции образуется новая переменная --- указатель на внешнюю переменную.

Сам указатель можно менять внутри функции.
Например, можно сделать чтобы он указывал на другую переменную.

Чтобы менять значение внешней переменной, нужно этот указатель
<<разыменовывать>>:

\begin{lstlisting}
  (*i) = 100;   
\end{lstlisting}

\end{frame}

\section{Ссылки и указатели в C++: общее и различия}

\begin{frame}[t,fragile]{C++ ссылка}
\textbf{Ссылка (reference)} --- это простой ссылочный тип, менее мощный, 
но более безопасный, чем указатель, унаследованный от языка C. 

С ссылкой нужно обращаться как с обычной переменной.
Например, её нельзя <<разыменовать>>.

Название <<C++ ссылка>> может приводить к путанице, 
так как в информатике под ссылкой понимается обобщенный концептуальный тип, 
а указатели и <<С++ ссылки>> являются специфическими реализациями ссылочного типа.
\end{frame}

\begin{frame}[t]{Отличия ссылки от указателя}
\begin{itemize}
\item Невозможно ссылаться напрямую на объект ссылочного типа после его определения; 
каждое упоминание его имени напрямую представляет объект, на который он ссылается.
\item При создании ссылки не могут быть выполнены никакие арифметические вычисления, приведения типов, взятия адреса и т.п.
\item После создания ссылки ее нельзя перевести на другой объект; в таких случаях говорят, не может быть \textbf{переопределена}. 
Это часто делают с указателями.
\item Ссылки не могут быть \texttt{NULL} (т.е. указывать в никуда), 
тогда как указатели - могут; 
каждая ссылка ссылается на некий объект, вне зависимости от его корректности.
\end{itemize}
\end{frame}

\begin{frame}[t,fragile]{Отличия ссылки от указателя...}
\begin{itemize}
\item Ссылки не могут быть неинициализированными. 
Так как невозможно переинициализировать ссылку, она должна быть инициализирована сразу после создания. 
В частности, локальные и глобальные переменные-ссылки должны быть проинициализированы там же, 
где они определены, а ссылки, которые являются данными-членами сущностей класса, 
должны быть инициализированы в списке инициализатора конструктора класса.
\end{itemize}
\begin{lstlisting}
 int& k; // компилятор выдаст сообщение: ошибка: 'k' declared as reference but not initialized 
         // ('k' объявлена как ссылка, но не инициализирована)
\end{lstlisting}
\end{frame}

\section{Структуры данных}

\begin{frame}[t,fragile]{Расстояние между 2-мя точками}

Точка на плоскости: задаётся 2-мя координатами: $(x; y)$

$dist(A, B) = \sqrt{{(A_x - B_x)^2} + {(A_y - B_y)^2}}$

\begin{lstlisting}
struct Point2D { double x, y; };

double dist(Point2D A, Point2D B) {
  return sqrt( pow(A.x - B.x, 2) + pow(A.y - B.y, 2));
}
\end{lstlisting}
\end{frame}

\begin{frame}[t,fragile]{Массивы}


\end{frame}
