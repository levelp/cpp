\section{Домашняя работа 2}

\subsection{Работа со структурами}

\begin{frame}[t,fragile]{Работа со структурами}

Дана структура данных - точка. И вектор - то же самое что и точка.
\begin{lstlisting}
struct Point { double x, y; };
typedef Point Vector;
\end{lstlisting}

Надо реализовать:
\begin{lstlisting}
double dist(Point a, Point b) { // Расстояние между точками
  return ... 
}
Point sum(Vector A, Vector B) { // Сумма векторов
}
Point sub(Vector A, Vector B) { // Разность векторов
}
double dotProduct(Vector A, Vector B) { // Скалярное произведение векторов
}
\end{lstlisting}
\end{frame}

\subsection{Скобочные последовательности}

\begin{frame}[t,fragile]{Аргумент: значение, ссылка, указатель}
\begin{lstlisting}
// По значению - by value - копируется значение
// Внутри функции возникает новая переменная i, при изменении которой
// "основная" i никак не меняется
void f1(int  i) { i++;     cout << "i = " << i << endl;  }
// По ссылке - by reference
// i внутри функции - новое имя (алиас, alias) для внешней переменной
void f2(int& i) { i++;     cout << "i = " << i << endl;  }
// По указателю - by pointer
void f3(int* i) { (*i)++;  cout << "i = " << *i << endl; }

int main() { // Основная программа
  int i = 3; // "Основная" переменная i
    // Сначала она равна 3
  f1(i);  cout << "i = " << i << endl; // i = 3 по-прежнему
  f2(i);  cout << "i = " << i << endl; // i увеличилась
  f3(&i); cout << "i = " << i << endl; // i ещё раз увеличилась
  return 0;
}
\end{lstlisting}
\end{frame}

\begin{frame}[t,fragile]{Аргумент: по значению - <<обычная>> переменная}
\begin{lstlisting}
// По значению - by value - копируется значение
// Внутри функции возникает новая переменная i, при изменении которой
// "основная" i никак не меняется
void f1(int  i) { 
  i++;    
  cout << "i = " << i << endl; 
}
\end{lstlisting}

Внутри функции появляется новая переменная с тем значением, которое было передано внутрь функции.

Эту переменную можно использовать и менять как угодно и это никак не отразится на 
внешней (вызывающей) программе.

Т.е. эта переменная только для внутреннего использования.  

\end{frame}

\begin{frame}[t,fragile]{Аргумент: по ссылке - только в С++, не в C}
\begin{lstlisting}
// По ссылке - by reference
// i внутри функции - новое имя (алиас, alias) для внешней переменной
void f2(int& i) { 
  i++;     
  cout << "i = " << i << endl;  
}
\end{lstlisting}

Внутри функции появляется переменная, которая является просто другим именем для
внешней переменной.

Если мы поменяем переменную внутри функции, изменится и внешняя переменная.

Это можно использовать чтобы возвращать значения из функции. 
Или если функция должна менять значения в основной программе.
Например, есть какой-то глобальный счётчик и функция должна его увеличивать и ещё что-то.
\end{frame}

\begin{frame}[t,fragile]{Аргумент: по указателю}
\begin{lstlisting}
// По указателю - by pointer
void f3(int* i) { 
  (*i)++;  
  cout << "i = " << *i << endl; 
}
\end{lstlisting}

При этом внутри функции образуется новая переменная --- указатель на внешнюю переменную.

Сам указатель можно менять внутри функции.
Например, можно сделать чтобы он указывал ну другую переменную.

Чтобы менять значение внешней переменной, нужно этот указатель
<<разыменовывать>>:

\begin{lstlisting}
  (*i) = 100;   
\end{lstlisting}

\end{frame}

\section{Структуры данных}

\begin{frame}[t,fragile]{Расстояние между 2-мя точками}

Точка на плоскости: задаётся 2-мя координатами: $(x; y)$

$dist(A, B) = \sqrt{{(A_x - B_x)^2} + {(A_y - B_y)^2}}$

\begin{lstlisting}
struct Point2D { double x, y; };

double dist(Point2D A, Point2D B) {
  return sqrt( pow(A.x - B.x, 2) + pow(A.y - B.y, 2));
}
\end{lstlisting}
\end{frame}

\subsection{Объединение (union)}

\begin{frame}[t,fragile]{Объединение (union)}

Структура данных, 
члены которой расположены по одному и тому же адресу.
 
Поэтому размер объединения равен размеру его наибольшего члена. 
В любой момент времени объединение хранит значение только одного из членов.

\begin{lstlisting}
union MyUnion {
  int i;
  double a;
}
\end{lstlisting}

Это объединение хранит либо целое число (переменная $i$), 
либо число с плавающей точкой (переменная $a$). 

Поскольку объединение --- вид структуры, то в 
\textbf{C/C++} к нему обращаются так же, 
как и к структуре: через символ 
\begin{lstlisting}
-> 
\end{lstlisting}
при использовании указателя, 
или <<.>> при использовании обычной переменной. 

Если мы обращаемся к $i$, то печатается целое число:

\begin{lstlisting}
  MyUnion m;
  cout << m.a << endl;
\end{lstlisting}
\end{frame}

\begin{frame}[t,fragile]{Пример использования union - байт по битам}
\begin{lstlisting}
union Bits { // Bits - название нашего типа
  unsigned char byte; // Байт
  struct byBits { // Байт по битам
    // Младший бит первый
    unsigned char b0 : 1, b1 : 1, b2 : 1, b3 : 1,
                  b4 : 1, b5 : 1, b6 : 1, b7 : 1;
  };
};
\end{lstlisting}

Если мы создадим переменную с типом Bits:
\begin{lstlisting}
Bits b;
b.byte = 3; // Значение байта 3 
cout << b.b0 << endl; // Печатаем нулевой бит
cout << b.b1 << endl; // Печатаем 1-ый бит
cout << b.b2 << endl; // Печатаем 2-ой бит
\end{lstlisting}
\end{frame}

\subsection{Union - представление чисел в памяти}
\begin{frame}[t,fragile]{Пример использования union - хранение в памяти}
  \lstinputlisting[language=C++, firstline=4]{02/06_show_in_memory_presentation/main.cpp}
\end{frame}


\section{Виды памяти}
\begin{frame}[t,fragile]{Виды памяти}

Память:   
\begin{itemize}
  \item Распределяемая компилятором (автоматически)
  \begin{itemize}
     \item Статическая --- существуют всё время работы программы, от начала до завершения
     \begin{itemize}
        \item Все глобальные переменные
        \item Все переменные с ключевым словом static
        \item Некоторые виды констант
     \end{itemize}
     \item Стек --- появляются при входе в функцию, удаляются при выходе из функции 
     \begin{itemize}
        \item Все локальные переменные (объявленные внутри функции)
        \item Параметры функции 
     \end{itemize}
  \end{itemize}
  \item Динамическая память, Heap, куча --- программист сам пишет когда память занять и освободить
     \begin{itemize}
        \item С: malloc / free + их варианты
        \item C++: new / delete
     \end{itemize}
  \end{itemize}
\end{frame}

\begin{frame}[t,fragile]{Статическая память и стек}

Как правило, статическая память больше стека.

\begin{lstlisting}
#include <iostream>
using namespace std;
const int SIZE = 100000000;

// Глобальный массив - будет в статической области памяти
int veryBigArray[SIZE];

void f(int n){  // n - в стеке
  int data[100]; // тоже в стеке
} // когда функция f завершится data будет автоматически удалён из стека
// и n тоже автоматически удалена из стека

int main() {
  // Он же объявленный внутри main - попадёт в стек
  int veryBigArray[SIZE]; // Stack overflow - переполнение стека
  static int i = 10; // i тоже попадёт в статическую область из-за static
  int j; // Переменная в стеке
  for(int k = 0; k < 10; k++) // k тоже в стеке
  return 0;
}
\end{lstlisting}  

При переполнении стека нужно:
     \begin{itemize}
        \item Увеличить стек настройками компилятора
        \item Переместить данные из стека в статическую или в динамическую память
     \end{itemize}
\end{frame}

\subsection{Динамическая область памяти (куча, heap)}
\begin{frame}[t,fragile]{Динамическая память в C:  / free}
Функция malloc() возвращает указатель на первый байт области памяти размером size, 
которая была выделена из динамически распределяемой области памяти. 
Если для удовлетворения запроса в динамически памяти нет достаточного объема памяти, 
возвращается нулевой указатель (NULL). 

Перед попыткой использовать выделенную память всегда проверяйте, 
что возвращаемое значение не является NULL. 
Попытка использовать нулевой указатель обычно приводит к завершению программы. 

\begin{lstlisting}  
  int* intArray; // Сам указатель находится в стеке
  intArray = (int*)malloc(100); // Получаем 100 байт в динамической памяти
  if(intArray == NULL) {
     printf("No memory!\n"); // Память закончилась
     return; // Выходим из функции
  }
  intArray[0] = 10; // Записываем значение в массив
  printf("intArray[0] = %d\n", intArray[0]); // Печатаем значение
  free(intArray); // Освобождаем память
\end{lstlisting}  

\end{frame}


\begin{frame}[t,fragile]{Динамическая память в С++: new / delete}

new --- оператор языка программирования C++, 
обеспечивающий выделение динамической памяти в куче. 

За исключением формы, называемой <<<размещающей формой new>>>
(можно менять поведение new), 
new пытается выделить достаточно памяти в куче для размещения новых данных и, 
в случае успеха, возвращает адрес свежевыделенной памяти. 

Если new не может выделить память в куче, то он передаст (throw) 
исключение типа std::bad\_alloc. 
Это устраняет необходимость явной проверки результата выделения. 

После встречи компилятором ключевого слова new им генерируется вызов конструктора класса.

Пусть у нас есть структура:
\begin{lstlisting}  
struct Point { double x, y; };
\end{lstlisting}  

\begin{lstlisting}  
  // Заводим массив в динамической памяти
  int* intArray; // Указатель на int
  intArray = new int[1000]; // new со скобками []
  intArray[0] = 10;
  intArray[1] = 23;
  delete[] intArray; // И delete должен быть со скобками

  Point* p = new Point; // new без []
  delete p; // delete без скобок []
\end{lstlisting}  

\end{frame}
